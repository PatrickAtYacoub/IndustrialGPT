\section{Einführung}
\label{sec:introduction}
\PARstart{I}{ndustrialGPT} repräsentiert einen bedeutenden Fortschritt in der Anwendung künstlicher Intelligenz im industriellen Sektor.
Dieses spezialisierte KI-Modell wurde entwickelt, um komplexe Herausforderungen in Produktionsumgebungen zu bewältigen.
Es zeichnet sich durch seine Fähigkeit aus, große Mengen industrieller Daten zu verarbeiten, Prozesse zu optimieren und Entscheidungsfindungen zu automatisieren.
IndustrialGPT verspricht, die Effizienz und Produktivität in verschiedenen Industriezweigen zu steigern, indem es fortschrittliche Datenanalyse, prädiktive Wartung und intelligente Automatisierung kombiniert.
Diese Studie untersucht die Potenziale und Anwendungsmöglichkeiten von IndustrialGPT sowie dessen Auswirkungen auf die zukünftige Gestaltung industrieller Prozesse.

\subsection{Vorgehen}
\label{subsec:approach}

Diese Arbeit untersucht die Entwicklung eines effizienten und qualitativ hochwertigen IndustrialGPT-Systems zur Informationsgewinnung aus PDFs im industriellen Kontext.
Zunächst wird die technische Architektur von IndustrialGPT erläutert.
Der Kern der Arbeit konzentriert sich auf die Implementierung und Evaluation fortschrittlicher Techniken wie zum Beispiel \textbf{Retrieval-Augmented Generation} (RAG), \textbf{Retrieval-Augmented Fine-tuning} (RAF), \textbf{In-Context Learning} (ICL), und \textbf{Few-Shot Learning}.
Diese Techniken werden genutzt, um die Fähigkeiten des Systems bei der Extraktion und Verarbeitung relevanter Informationen aus industriellen Dokumenten zu verbessern.
Die Arbeit evaluiert auch den Einsatz von Prompt Engineering und Chain-of-Thought-Prompting zur Optimierung der Leistung.
Die Leistung des entwickelten IndustrialGPT wird anschließend mit einem bestehenden System verglichen, das auf GPT-3.5 basiert.
Die Arbeit bewertet die Effizienz, Genauigkeit und Anwendbarkeit beider Systeme in industriellen Szenarien und diskutiert die Implikationen für zukünftige Entwicklungen im Bereich der industriellen KI-Anwendungen.
